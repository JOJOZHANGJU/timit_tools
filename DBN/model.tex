\documentclass[a4paper,12pt]{article}

\usepackage[utf8]{inputenc}
\usepackage[T1]{fontenc} % serif: fontenc, palatino, fouriernc, lmodern
\usepackage{fullpage}
\usepackage{amsfonts, amsmath, amssymb, amstext, latexsym, stmaryrd, ulem}
\usepackage{graphicx, epsfig, wrapfig, exscale, amsbsy, amsopn, fancyhdr}
\usepackage{subfigure, xspace, enumerate, caption, mathrsfs, pdfpages}
\usepackage{url, tikz}
\usetikzlibrary{positioning, arrows}


\begin{document}

\begin{center}
{\fontsize{20pt}{20pt} \selectfont \textbf{Deep learning models for speech}}
\vskip 10pt
Gabriel Synnaeve
\vskip 10pt
\hrule height 4pt
\end{center}

This presents some of the models found here.

\section{Basic HMM-DBN}

%%% TODO

\section{Sequential Stacked DBN}

%%% TODO
% Stacked Sequential Learning.
% Parameters: a history size Wh , a future size Wf , and a cross-validation parameter K.
% Learning algorithm: Given a sample S = {(xt , yt )}, and a sequential learning algorithm A:
% 1. Construct a sample of predictions yˆt for each xt ∈ S as follows:
% (a) Split S into K equal-sized disjoint subsets S1,...,SK
% (b) Forj=1,...,K,letfj =A(S−Sj)
% (c) LetSˆ={(xt,yˆt):yˆt =fj(xt)andxt ∈Sj}
% 2. Construct an extended dataset S′ of instances (x′t, yt) by converting each xt to x′t as follows: xt′ = ⟨x′1,...,x′lt⟩ where x′i = (xi,yˆi−Wh,...,yˆi+Wf ) and yˆi is the i-th component of yˆt, the label vector paired with xt in Sˆ.
% 3. Return two functions: f = A(S) and f′ = A(S′).
% Inference algorithm: given an instance vector x:
% 1. Letyˆ=f(x)
% 2. Carry out Step 2 above to produce an extended in- stance x′ (using yˆ in place of yˆt).
% 3. Return f′(x′).

\section{RNN}

Activation $a_i(t)$ for layer $i$ at time $t$ is:
\begin{equation}
    a_i(t) = \tanh (W_i a_i(t-1) + Z_i a_{i-1}(t) + b_{i-1}(t))
\end{equation}
With $a_0 + b_0$ being the signal (input features).

\section{RNN speech + speaker}

Alternate between backpropagation with:

\begin{eqnarray}
    loss(speaker, y_{spkr}) \\
    loss(phone, y_{phn})  
\end{eqnarray}


\section{Embeddings-based loss functions}

%%% TODO


\end{document}
